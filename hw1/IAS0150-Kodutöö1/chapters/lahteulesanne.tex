Loogikaülesanne luuakse matrikli \studentcode{} järgi.

\begin{table}[!ht]
\centering
\begin{tabular}{|l l l|}
 \hline
 fn & ühed & määramatused \\
 \hline
 1: & \tt{20CE0223} & \tt{AEF560B} \\
 2: & \tt{19345B97} & \tt{866C932} \\
 3: & \tt{99B1AD47} & \tt{333B39C2} \\
 4: & \tt{171022C3} & \tt{7B00B96} \\
 \hline
\end{tabular}
\caption{Ühtede ja määramatuste piirkonnad funktsioonides}
\label{table:1}
\end{table}

Tabeli \ref{table:1} abil saadakse järgmine funktsioonide süsteem:
\[\begin{array}{l}
f1(x_1, x_2, x_3, x_4) = \sum(0,2,3,12,14)_1 (5,6,10,11,15)_\_ \\
f2(x_1, x_2, x_3, x_4) = \sum(1,3,4,5,7,9,11)_1 (2,6,8,12)_\_ \\
f3(x_1, x_2, x_3, x_4) = \sum(1,4,7,9,10,11,13)_1 (2,3,12)_\_ \\
f4(x_1, x_2, x_3, x_4) = \sum(0,1,2,3,7,12)_1 (6,9,11)_\_ \\
\end{array}\]
Samaväärne tõeväärtustabel on kuvatud tabelis \ref{table:2}
\begin{table}[ht!]
    \centering
    \begin{tabular}{|c|c c c c|c c c c|}
            \hline
            & \(x_1\)&\(x_2\)&\(x_3\)&\(x_4\) & \(f1\)&\(f2\)&\(f3\)&\(f4\)\\
            \hline
            0 & 0&0&0&0 & 1&0&0&0\\
            1 & 0&0&0&1 & 0&1&1&1\\
            2 & 0&0&1&0 & 1&-&-&1\\
            3 & 0&0&1&1 & 1&1&-&1\\
            \hline
            4 & 0&1&0&0 & 0&1&1&0\\
            5 & 0&1&0&1 & -&1&0&0\\
            6 & 0&1&1&0 & -&-&0&-\\
            7 & 0&1&1&1 & 0&1&1&1\\
            \hline
            8 & 1&0&0&0 & 0&-&0&0\\
            9 & 1&0&0&1 & 0&1&1&-\\
            A & 1&0&1&0 & -&0&1&0\\
            B & 1&0&1&1 & -&1&1&-\\
            \hline
            C & 1&1&0&0 & 1&-&-&1\\
            D & 1&1&0&1 & 0&0&1&0\\
            E & 1&1&1&0 & 1&0&0&0\\
            F & 1&1&1&1 & -&0&0&0\\
            \hline
    \end{tabular}
    \caption{Funktsioonide süsteem tõeväärtustabeli kujul}
    \label{table:2}
\end{table}{}